\documentclass{article}
\usepackage[final, nonatbib]{neurips_2024}
\usepackage{amsmath, amssymb}
\usepackage{graphicx}
\usepackage{booktabs}
\usepackage{natbib}
\usepackage{hyperref}
\usepackage{url}
\bibliographystyle{unsrtnat}

\title{NBA Shot Distribution \& Winning: A Bayesian Model}
\author{
  Ethan Babel \\
  Johns Hopkins University \\
  \texttt{ebabel1@jh.edu}
}

\begin{document}
\makeatletter
\renewcommand{\@noticestring}{}
\makeatother

\maketitle

\begin{abstract}
      Understanding how shot selection contributes to winning has become a central
      theme in modern basketball analytics, especially as the NBA has shifted
      toward a more prominent three point focus offensively over the past two decades. 
      In this project, I study how the distribution of field-goal attempts a team both 
      takes on offense and concedes on defense across standard distance bands affects 
      the team's probability of winning games. I develop a Bayesian logistic regression 
      model that uses each team's offensive and defensive shot-distance distributions as 
      predictors. Posterior inference is done via Metropolis–Hastings sampling,
      yielding full distributions over the effects of each shot band and enabling 
      comparisons across eras.

      \medskip

      I apply the model to two NBA seasons separated by twenty years (2004-05 and
      2024-25) to evaluate how the value of shot distribution has evolved over time.
      The model achieves substantially higher predictive skill in the modern season
      (Brier Skill Score: 0.26) compared to the earlier era (0.04), suggesting that
      shot selection has become a far more impactful determinant of success in today's
      spacing and efficiency driven league. Key findings indicate that in 2004-05, teams that
      take a greater share of their shots from three-point range and the rim while limiting
      the same areas defensively tend to win significantly more games. By 2024-25, these effects
      lessened or even reversed, suggesting that teams have adapted to prioritize three-point 
      shooting more universally, and indicating that some teams today may benefit from
      diversifying their shot profiles back toward mid-range attempts. 
\end{abstract}

\section{Introduction}

The past two decades have seen a dramatic transformation in how NBA teams
structure their offenses. Long mid-range jumpers have steadily declined, while
three-point attempts and shots at the rim have risen sharply. These changes are
well documented in the basketball analytics literature and are often attributed
to the growing influence of efficiency-based decision making, where the expected
value of each shot type guides offensive schemes. Although this historical trend
is clear, this project delves into determining to what extent a team’s shot 
distribution can actually predict its success, as well as comparing the 
importance of shot distribution across eras.

I address these questions by developing a Bayesian model that links
team-level win probabilities to offensive and defensive shot-distance
distributions. Distance-based shot charts (such as the five standard NBA
zones 0--3 ft, 3--10 ft, 10--16 ft, 16 ft -- 3 point line, and three-pointers) 
provide a compact representation of a team's spatial tendencies. Teams differ not 
only in where they choose to shoot but also in the types of shots they allow
opponents to take. A comprehensive analysis therefore requires modeling both
dimensions.

While there are undoubtedly far more factors influencing wins (e.g., shooting
efficiency/ability, turnovers, rebounding, pace, etc.), this study isolates 
the role of shot distribution to understand its standalone predictive power.
I use a binomial logistic regression model in which the log-odds of winning are
determined by a linear combination of the proportion of shots taken/allowed by 
a team each offensive and defensive shot band. Because the posterior distribution 
of the model parameters is not available in closed form, I employ a Metropolis–Hastings 
sampler to obtain posterior draws, enabling full uncertainty quantification and posterior
predictive evaluation. This Bayesian framework offers advantages over classical
approaches, mainly that it provides interpretable posterior distributions for each shot
band’s effect, naturally incorporates parameter uncertainty into predictive
estimates, and allows comparison of structural effects across seasons.

The analysis focuses on two seasons separated by twenty years: 2004--05 and
2024--25. This time span captures both the pre–three-point-revolution era and the
fully modern, spacing-oriented NBA. Comparing results across these seasons
allows us to quantify whether shot distribution has become more (or less)
predictive of success over time. 

Overall, this study provides a Bayesian perspective on the evolution of shot
selection in professional basketball, revealing how the value of different shot
types has shifted and how both offensive and defensive spatial tendencies
jointly shape team performance.

\section{Related Works}

A substantial body of research examines how the spatial distribution of shots,
rather than merely shooting efficiency, relates to team success in modern
basketball. This literature can be divided into three major themes:
(1) long term evolution of shot patterns, and (2) the rise of the three-point shot,
the decline of the mid-range, and the effects of these trends on winning.

\subsection{Evolution of NBA Shot Distribution}

Wang and Zheng~\citep{wang2022changes} document league-wide trends in shooting
patterns over the past decade, showing sharp declines in long mid-range
attempts and concurrent increases in three-point volume. They also note that
offensive and defensive field-goal percentages remain the strongest predictors
of winning, linking shot location to macro-level team outcomes.

Zajac et al.~\citep{zajac2023longterm} expand this view by studying forty
consecutive NBA seasons. They find rising efficiency and volume from three, as
well as stability in two-point accuracy, concluding that effective field-goal
percentage (eFG\%), a metric that adjusts traditional field goal percentage by 
fairly weighting 3-point attempts overwhelmingly determines win probability. 
\[
      eFG\% = \frac{FGM + 0.5 \times 3PM}{FGA}
\]
These results reinforce the idea that spatial shot tendencies shape league-wide 
offensive performance.

\subsection{Three-Point Revolution and Decline of Mid-Range Shooting}

Kilcoyne~\citep{kilcoyne2020midrange} analyzes the decline of mid-range
attempts in the context of basketball analytics. He finds that mid-range
volume is negatively associated with team wins once efficiency is accounted
for, supporting the common interpretation that these shots are dominated by
higher-value alternatives (rim attempts or three-pointers).

In contrast, Winn~\citep{winn2023threepoint} shows that three-point attempt rate
is not always a statistically significant predictor of team success once other
factors are controlled for. This highlights that volume alone is insufficient
without context.

\subsection{Positioning of This Work}

The literature demonstrates that NBA shot distribution has changed dramatically 
over time, and that winning correlates more strongly with efficiency than raw volume.

This project adds to the existing literature by modeling team-level win
probabilities directly as a function of \emph{both} offensive and defensive 
shot-distance distributions using a Bayesian logistic framework. The Bayesian 
approach improves on the works above  by providing full posterior uncertainty 
quantification for each shot band’s effect, enabling richer interpretation and
comparison across eras.

\section{Model}

\subsection{Data and notation}

We analyze two NBA regular seasons, $s \in \{2004\text{--}05,\,2024\text{--}25\}$.
For each season $s$ we observe $n_s = 30$ teams indexed by $i = 1,\dots,n_s$.
Let
\[
y_{is} \in \{0,1,\dots,N\}
\]
denote the number of regular-season wins for team $i$ in season $s$, where
$N = 82$ is the number of regular-season games.

For each team we summarize its offensive and defensive shot distribution using
the proportion of field-goal attempts (FGA) taken or allowed from four distance
bands:
\begin{align*}
\text{Offense:}\quad &
x^{\text{off}}_{is,0\text{--}3},
\;
x^{\text{off}}_{is,3\text{--}10},
\;
x^{\text{off}}_{is,10\text{--}16},
\;
x^{\text{off}}_{is,16\text{--}3P},\\[4pt]
\text{Defense:}\quad &
x^{\text{def}}_{is,0\text{--}3},
\;
x^{\text{def}}_{is,3\text{--}10},
\;
x^{\text{def}}_{is,10\text{--}16},
\;
x^{\text{def}}_{is,16\text{--}3P}.
\end{align*}
These correspond respectively to $0$--$3$ ft, $3$--$10$ ft, $10$--$16$ ft, and
$16$ ft to the three-point line. The offensive and defensive three-point
shares,
\[
x^{\text{off}}_{is,3P}, \quad x^{\text{def}}_{is,3P},
\]
are omitted from the regression design matrix to avoid perfect collinearity,
but are later reconstructed from the results of the included features for 
diagnostic and interpretive calculations.

For season $s$, we collect the eight included shot-share features into a
vector
\[
\mathbf{x}_{is}
=
\Big(
x^{\text{off}}_{is,0\text{--}3},
x^{\text{off}}_{is,3\text{--}10},
x^{\text{off}}_{is,10\text{--}16},
x^{\text{off}}_{is,16\text{--}3P},
x^{\text{def}}_{is,0\text{--}3},
x^{\text{def}}_{is,3\text{--}10},
x^{\text{def}}_{is,10\text{--}16},
x^{\text{def}}_{is,16\text{--}3P}
\Big)^\top \in \mathbb{R}^8,
\]
so that the $(i,s)$ design row is $\mathbf{x}_{is}$ and the response is $y_{is}$.

\subsection{Standardization of predictors}

Before fitting, we standardize each feature dimension within a season.  Let
$\mu_{s,j}$ and $\sigma_{s,j}$ denote the empirical mean and standard deviation
of the $j$-th feature across teams in season $s$:
\[
\mu_{s,j} = \frac{1}{n_s} \sum_{i=1}^{n_s} x_{is,j}, 
\qquad
\sigma_{s,j}
= \sqrt{\frac{1}{n_s} \sum_{i=1}^{n_s} \bigl(x_{is,j} - \mu_{s,j}\bigr)^2}.
\]
We then define standardized covariates
\[
z_{is,j} = \frac{x_{is,j} - \mu_{s,j}}{\sigma_{s,j}}, \qquad j = 1,\dots,8.
\]
In matrix notation we write
\[
\mathbf{z}_{is} = \bigl(z_{is,1},\dots,z_{is,8}\bigr)^\top,
\]
and stack these into a season-specific design matrix
$Z_s \in \mathbb{R}^{n_s \times 8}$.

\subsection{Likelihood}

We model team wins via a Binomial likelihood with a logistic link.  For each
team $i$ in season $s$,
\begin{equation}
  y_{is} \mid p_{is} \sim \text{Binomial}(N, p_{is}),
  \label{eq:binom-like}
\end{equation}
where $p_{is} \in (0,1)$ is the underlying win probability for team $i$.

The logit of $p_{is}$ is modeled as a linear function of the standardized
offensive and defensive shot shares:
\begin{equation}
  \text{logit}(p_{is})
  \;=\;
  \alpha_s + \mathbf{z}_{is}^\top \boldsymbol{\beta}_s
  \;=\;
  \alpha_s + \sum_{j=1}^{8} \beta_{s,j} z_{is,j},
  \label{eq:logit-linear}
\end{equation}
where $\alpha_s$ is a season-specific intercept and
$\boldsymbol{\beta}_s = (\beta_{s,1},\dots,\beta_{s,8})^\top$ are the
season-specific slope coefficients for the eight included bands (four offensive
and four defensive).

Rearranging \eqref{eq:logit-linear}, we arrive at an expression for the win
probability:
\begin{equation}
  p_{is}
  =
  \sigma\!\bigl(
    \alpha_s + \mathbf{z}_{is}^\top \boldsymbol{\beta}_s
  \bigr)
  =
  \frac{
    \exp\bigl(
      \alpha_s + \mathbf{z}_{is}^\top \boldsymbol{\beta}_s
    \bigr)
  }{
    1 + \exp\bigl(
      \alpha_s + \mathbf{z}_{is}^\top \boldsymbol{\beta}_s
    \bigr)
  },
  \label{eq:win-prob}
\end{equation}
where $\sigma(\cdot)$ denotes the logistic sigmoid function.

Letting
$\boldsymbol{\theta}_s = (\alpha_s,\boldsymbol{\beta}_s)^\top \in \mathbb{R}^9$,
the season-$s$ log-likelihood for all teams is
\begin{equation}
  \ell_s(\boldsymbol{\theta}_s)
  =
  \sum_{i=1}^{n_s}
  \Bigl[
    y_{is} \log p_{is}(\boldsymbol{\theta}_s)
    +
    (N - y_{is}) \log (1 - p_{is}(\boldsymbol{\theta}_s))
  \Bigr],
  \label{eq:loglik}
\end{equation}
where $p_{is}(\boldsymbol{\theta}_s)$ is given by \eqref{eq:win-prob}.

\subsection{Priors}

We place independent Normal priors on the intercept and slope coefficients for
each season.  For the intercept,
\begin{equation}
  \alpha_s \sim \mathcal{N}(0, \sigma_\alpha^2),
  \qquad
  \sigma_\alpha = 0.5,
  \label{eq:prior-alpha}
\end{equation}
and for each slope coefficient,
\begin{equation}
  \beta_{s,j} \sim \mathcal{N}(0, \sigma_\beta^2),
  \qquad
  \sigma_\beta = 0.1,
  \qquad
  j = 1,\dots,8.
  \label{eq:prior-betas}
\end{equation}
These priors are weakly informative on the standardized scale, encoding the
prior belief that:
\begin{enumerate}
      \item The league-average team has roughly 50\%
      win probability.

      \item A one-standard-deviation change in any shot-share
      feature is unlikely to change the log-odds of winning by more than roughly
      $\pm 0.2$ to $\pm 0.3$.
\end{enumerate}


While the parameters undoubtedly have some correlation (particularly between parameters
on the same side of the ball), we assume independence for simplicity reasons. Thus the 
log-prior density for $\boldsymbol{\theta}_s$ is
\begin{equation}
  \log \pi(\boldsymbol{\theta}_s)
  =
  \log \mathcal{N}(\alpha_s \mid 0, \sigma_\alpha^2)
  + \sum_{j=1}^{8} \log \mathcal{N}(\beta_{s,j} \mid 0, \sigma_\beta^2).
  \label{eq:log-prior}
\end{equation}

\subsection{Posterior and MCMC sampling}

Combining \eqref{eq:loglik} and \eqref{eq:log-prior}, the unnormalized log-posterior
for season $s$ is
\begin{equation}
  \log \pi(\boldsymbol{\theta}_s \mid \{y_{is},\mathbf{z}_{is}\})
  =
  \ell_s(\boldsymbol{\theta}_s)
  + \log \pi(\boldsymbol{\theta}_s).
  \label{eq:log-post}
\end{equation}
Because of the logistic link and normal priors, the model is non-conjugate and there
is no closed-form expression for the posterior distribution of
$\boldsymbol{\theta}_s$.

We therefore approximate the posterior via Markov chain Monte Carlo (MCMC),
using the random-walk Metropolis algorithm.  For each season $s$ we run a
single-chain sampleras follows:

\begin{itemize}
  \item Number of iterations:
        $T = 105{,}000$ total proposals.
  \item Burn-in:
        the first $25{,}000$ iterations are discarded.
  \item Proposal distribution:
        for the current state $\boldsymbol{\theta}$,
        \[
        \boldsymbol{\theta}' = \boldsymbol{\theta} + \boldsymbol{\epsilon},
        \qquad
        \boldsymbol{\epsilon} \sim \mathcal{N}\bigl(\mathbf{0}, \Sigma_{\text{prop}}\bigr),
        \]
        with $\Sigma_{\text{prop}} = \text{diag}(s_0^2, s_1^2, \dots, s_8^2)$,
        where $s_0 = 0.05$ for the intercept and $s_j = 0.05$ for all slopes.
  \item Acceptance probability:
        with $\log \pi(\cdot)$ given by \eqref{eq:log-post}, we accept
        the proposal with probability
        \[
        \alpha(\boldsymbol{\theta},\boldsymbol{\theta}')
        =
        \min\!\left\{
          1,\;
          \exp\bigl[
            \log \pi(\boldsymbol{\theta}') -
            \log \pi(\boldsymbol{\theta})
          \bigr]
        \right\}
        =
        \min\!\left\{
          1,\;
          \frac{\boldsymbol{\theta}'}{\boldsymbol{\theta}}
        \right\}.
        \]
\end{itemize}

We initialize the chain at $\boldsymbol{\theta}^{(0)} = \mathbf{0}$ and fix
the random seed to facilitate reproducibility.  For each season we record the
acceptance rate and the trace of the log-posterior values
$\log \pi(\boldsymbol{\theta}^{(t)})$.

\subsection{MCMC diagnostics and effective sample size}

To assess convergence and mixing of the random-walk Metropolis sampler, we
examine trace plots, autocorrelation functions (ACFs), and effective sample
size (ESS) estimates for all parameters. Figure \ref{fig:trace-alpha-both} shows
representative trace and ACF plots for the parameter corresponding to the 0--3 ft 
distance in the 2004--05 and 2024--25 seasons. The chain appears well-mixed, with 
no visible nonstationarity and rapidly decaying autocorrelation.

\begin{figure}[htbp]
    \centering
    \begin{minipage}{0.48\linewidth}
        \centering
        \includegraphics[width=\linewidth]{../output/trace_acf/2004-05_0-3_trace_acf.png}
    \end{minipage}
    \hfill
    \begin{minipage}{0.48\linewidth}
        \centering
        \includegraphics[width=\linewidth]{../output/trace_acf/2024-25_0-3_trace_acf.png}
    \end{minipage}
    \caption{Trace and autocorrelation for the 0--3 ft parameter ($\beta_1$) for both seasons.}
    \label{fig:trace-alpha-both}
\end{figure}

Effective sample size (ESS) values for the 2004--05 and 2024--25
seasons are shown in Table ~\ref{tab:ess-both-seasons}. In
both seasons, ESS values are sufficiently large (near a thousand or more), 
indicating that the posterior summaries are reliable.

\begin{table}[htbp]
\centering
\caption{ESS for 2004--05 and 2024--25 seasons.}
\label{tab:ess-both-seasons}

\begin{tabular}{l cc}
\toprule
& \multicolumn{2}{c}{\textbf{Effective Sample Size (ESS)}} \\
\cmidrule(lr){2-3}
\textbf{Parameter} 
& \textbf{2004--05} 
& \textbf{2024--25} \\
\midrule
$\alpha$         & 1798.7 & 2113.3 \\
$0$--$3$ (off)   & 1094.2 & 918.8  \\
$3$--$10$ (off)  & 1089.0 & 1166.2 \\
$10$--$16$ (off) & 980.9  & 1365.5 \\
$16$--3P (off)   & 1166.5 & 1119.2 \\
$0$--$3$ (def)   & 809.8  & 704.9  \\
$3$--$10$ (def)  & 902.2  & 797.9  \\
$10$--$16$ (def) & 1160.0 & 1462.9 \\
$16$--3P (def)   & 946.4  & 1264.0 \\
\bottomrule
\end{tabular}
\end{table}

\subsection{Posterior predictive fit and residual diagnostics}

Posterior mean win probabilities $\hat{p}_{is}$ provide a simple summary of
model fit. Figure ~\ref{fig:pred-vs-actual-side} compares posterior predicted
win probabilities with observed "true" win rates for both seasons. The closer points
lie to the 45-degree line, the better calibrated the model is. The 2024--25
season exhibits noticeably stronger predictive alignment, consistent with the
higher Brier skill score.

\begin{figure}[htbp]
    \centering
    \begin{minipage}{0.48\linewidth}
        \centering
        \includegraphics[width=\linewidth]{../output/plots/2004-05_pred_vs_actual.png}
    \end{minipage}
    \hfill
    \begin{minipage}{0.48\linewidth}
        \centering
        \includegraphics[width=\linewidth]{../output/plots/2024-25_pred_vs_actual.png}
    \end{minipage}
    \caption{Posterior mean predicted win probability vs.\ actual win rate for
    both seasons.}
    \label{fig:pred-vs-actual-side}
\end{figure}

Residual diagnostics evaluate whether the model suffers from systematic bias. 
For each team we compute residuals
\[
r_{is} = \frac{y_{is}}{N} - \hat{p}_{is}.
\]
Sample residual plots are shown in Figure~\ref{fig:residual-side}, which displays
residuals against the 0--3 ft offensive shot share for both seasons. No clear
patterns of bias are visible, suggesting that the model adequately captures the
relationship between shot distribution and winning. 

\begin{figure}[htbp]
    \centering
    \begin{minipage}{0.48\linewidth}
        \centering
        \includegraphics[width=\linewidth]{../output/residuals_analysis/2004-05_residuals_0-3.png}
    \end{minipage}
    \hfill
    \begin{minipage}{0.48\linewidth}
        \centering
        \includegraphics[width=\linewidth]{../output/residuals_analysis/2024-25_residuals_0-3.png}
    \end{minipage}
    \caption{Residuals vs.\ 0--3 ft offensive shot share for both seasons.}
    \label{fig:residual-side}
\end{figure}

\subsection{Cross-validated performance}

To assess out-of-sample predictive accuracy, we perform leave-one-out
cross-validation within each season. For each team $i$ we refit the
model on the remaining $n_s - 1$ teams and compute the posterior mean predicted
win probability $\hat{p}^{(-i)}_{is}$. Predictive accuracy is quantified using
the Brier score,
\[
\text{Brier}_s = \frac{1}{n_s} \sum_{i=1}^{n_s}
\Bigl( \hat{p}^{(-i)}_{is} - \frac{y_{is}}{N} \Bigr)^2,
\]
and the Brier skill score, which compares the model to a baseline predictor
that simply predicts the in-sample average win rate:
\[
\text{Skill}_s
= 1 - \frac{\text{Brier}_s}{\text{Brier}_{\text{baseline}, s}}.
\]

Table~\ref{tab:brier-summary-both} reports the cross-validated Brier scores and
skill scores for both seasons. The 2024--25 model demonstrates markedly higher
predictive skill, indicating that offensive and defensive shot distribution
explains substantially more variation in team success in the modern NBA than in
the earlier era.

\begin{table}[htbp]
\centering
\caption{Cross-validated Brier score summaries for both seasons.}
\label{tab:brier-summary-both}

\begin{tabular}{l cc}
\toprule
\textbf{Metric} 
& \textbf{2004--05 Season} 
& \textbf{2024--25 Season} \\
\midrule
Brier score              & 0.0237 & 0.0203 \\
Baseline Brier score     & 0.0248 & 0.0274 \\
Brier skill score        & 0.0425 & 0.2611 \\
\bottomrule
\end{tabular}
\end{table}

\section{Results}

This section presents the estimated posterior effects of offensive and 
defensive shot distribution on team success. Figure ~\ref{fig:post-forest-side} 
displays the posterior mean and 95\% credible interval for every shot-distance 
band are shown in the forest plots below. All effects are interpreted 
on the log-odds scale per one standard-deviation change in shot share.
Note that the 3P and def\_3P effects represent the implied contributions 
of the omitted compositional bucket.

\begin{figure}[htbp]
    \centering
    \begin{minipage}{0.48\linewidth}
        \centering
        \includegraphics[width=\linewidth]{../output/plots/2004-05_posterior_forest.png}
    \end{minipage}
    \hfill
    \begin{minipage}{0.48\linewidth}
        \centering
        \includegraphics[width=\linewidth]{../output/plots/2024-25_posterior_forest.png}
    \end{minipage}
    \caption{Posterior effects of shot distribution on log-odds of winning (per SD change).}
    \label{fig:post-forest-side}
\end{figure}

\subsection{2004--05 Season}

Offensively, it is clear that three-point have a strong positive association with winning. 
Mid-range bands (3--10, 10--16, 16--3P) on the other hand generally show neutral or negative 
effects. Moreover, the graph shows that preventing opponents from shooting in the 
effecient 3P zone was also a major separator between winning and losing teams, with conceding 
three-point attempts being by far the largest defensive detriment.

One notable observation for this season is the slightly negative effect for offensive rim 
attempts (0--3 ft). While counterintuitive at first glance, this likely reflects that teams
in this era took an excessive share of their shots at the rim. The linear nature of the model
does not capture diminishing returns, so the overwhelming number of teams taking a high proportion
of rim attempts due to the style of play may have led to this negative association.

\subsection{2024--25 Season}

In contrast to what one might expect, the three-point effect in 2024--25 weakens significantly, 
with the interval mean being less than zero, indicating that aking more threes than league 
average no longer provides the same marginal advantage. Additionally, several mid-range bands 
now exhibit less negative or even slightly positive effects. The value of offensive rim attempts
also explodes positively, suggesting that in the modern era where teams space the floor far
more effectively, attacking the rim has become a more potent weapon. This result also somewhat
validates the provided explanation for the negative rim effect in 2004--05. namely that the
linear constraint of the model prevented capturing diminishing returns in an era where rim 
attempts were overused. These patterns suggest that as the league has converged on analytically 
optimized shot selection, marginal returns to additional three-point volume have diminished, and
some teams may even benefit from reintroducing a more diversified shot profile.

Defensively however, the effect of conceding large quantities of three-point attempts 
remains negative, even exceeding the impact seen in 2004--05. This underscores that even if 
taking more threes may not be as advantageous as before, preventing opponents from doing 
so is critical. With defending the rim also as important as ever, it is clear that while the 
optimal offensive shot profile is still debatable, modern defensive schemes must prioritize
protecting the rim and the perimeter, while funneling opponents into mid-range areas.

\subsection{Interpretation Summary}

Overall, these results show that in 2004--05, exploiting high-efficiency zones (rim and three) 
created substantial competitive differentiation.By 2024--25, league-wide strategic convergence 
has flattened these advantages. Modern teams may gain edges through spacing that re-opens rim 
opportunities, retaining credible mid-range threats, and implementing defensive schemes emphasizing 
rim and perimeter protection while encouraging the mid-range attempts that increasingly unfamiliar
to today's players.

\section{Discussion}

\noindent\textbf{Limitations.}
Although informative, this analysis is subject to several important limitations:
\begin{itemize}
    \item The model incorporates only shot-distance distribution, not shot efficiency 
    (FG\%, eFG\%, TS\%), and thus cannot distinguish whether teams are good at generating 
    or defending specific shots.
    \item The linear structure cannot capture nonlinear "sweet spots" or diminishing 
    returns for particular shot types.
    \item Pace, possessions, opponent strength, and other contextual variables are not included.
    \item Only two seasons are analyzed, providing a limited sample for inference and 
    reducing generalizability.
    \item Potentially important factors such as coaching, injuries, roster changes, 
    and defensive scheme are not modeled.
\end{itemize}

\noindent\textbf{Future Improvements.}
Several extensions could strengthen and broaden the scope of this work:
\begin{itemize}
    \item Incorporate zone-specific shooting efficiency metrics and additional team 
    performance indicators such as pace, turnover rate, rebound rate, and free-throw rate.
    \item Extend the hierarchical framework using many seasons to jointly estimate 
    league-wide and season-specific effects.
    \item Move from season-level aggregates to game-by-game modeling to increase sample 
    size and control for opponent-adjusted factors.
    \item Evaluate posterior predictive performance by holding out entire games or entire seasons.
    \item Introduce team-specific random effects to account for unobserved individual 
    circumstances (e.g., coaching strategies or roster stability).
\end{itemize}

All code used in this analysis, including data processing pipelines, MCMC implementation,
and visualization scripts, is available in the project repository: 

\url{https://github.com/ethanbabel/Bayesian-Stats-Final-Project}.

\bibliography{references}

\end{document}